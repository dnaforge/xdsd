%% Generated by Sphinx.
\def\sphinxdocclass{report}
\documentclass[a4paper,12pt,oneside,english]{sphinxmanual}
\ifdefined\pdfpxdimen
   \let\sphinxpxdimen\pdfpxdimen\else\newdimen\sphinxpxdimen
\fi \sphinxpxdimen=.75bp\relax

\usepackage[utf8]{inputenc}
\ifdefined\DeclareUnicodeCharacter
 \ifdefined\DeclareUnicodeCharacterAsOptional
  \DeclareUnicodeCharacter{"00A0}{\nobreakspace}
  \DeclareUnicodeCharacter{"2500}{\sphinxunichar{2500}}
  \DeclareUnicodeCharacter{"2502}{\sphinxunichar{2502}}
  \DeclareUnicodeCharacter{"2514}{\sphinxunichar{2514}}
  \DeclareUnicodeCharacter{"251C}{\sphinxunichar{251C}}
  \DeclareUnicodeCharacter{"2572}{\textbackslash}
 \else
  \DeclareUnicodeCharacter{00A0}{\nobreakspace}
  \DeclareUnicodeCharacter{2500}{\sphinxunichar{2500}}
  \DeclareUnicodeCharacter{2502}{\sphinxunichar{2502}}
  \DeclareUnicodeCharacter{2514}{\sphinxunichar{2514}}
  \DeclareUnicodeCharacter{251C}{\sphinxunichar{251C}}
  \DeclareUnicodeCharacter{2572}{\textbackslash}
 \fi
\fi
\usepackage{cmap}
\usepackage[T1]{fontenc}
\usepackage{amsmath,amssymb,amstext}
\usepackage[english]{babel}
\usepackage{mathpazo}
\usepackage[Bjarne]{fncychap}
\usepackage[dontkeepoldnames]{sphinx}

\usepackage{geometry}

% Include hyperref last.
\usepackage{hyperref}
% Fix anchor placement for figures with captions.
\usepackage{hypcap}% it must be loaded after hyperref.
% Set up styles of URL: it should be placed after hyperref.
\urlstyle{same}
\addto\captionsenglish{\renewcommand{\contentsname}{Contents:}}

\addto\captionsenglish{\renewcommand{\figurename}{Fig.}}
\addto\captionsenglish{\renewcommand{\tablename}{Table}}
\addto\captionsenglish{\renewcommand{\literalblockname}{Listing}}

\addto\captionsenglish{\renewcommand{\literalblockcontinuedname}{continued from previous page}}
\addto\captionsenglish{\renewcommand{\literalblockcontinuesname}{continues on next page}}

\addto\extrasenglish{\def\pageautorefname{page}}

\setcounter{tocdepth}{3}
\setcounter{secnumdepth}{3}


\title{XDSD}
\date{Sep 18, 2022}
\release{0.1}
\author{}
\newcommand{\sphinxlogo}{\sphinxincludegraphics{XDSD_UserManual_cover.pdf}\par}
\renewcommand{\releasename}{Release}
\makeindex

\begin{document}

\maketitle
\sphinxtableofcontents
\phantomsection\label{\detokenize{index::doc}}


DSDui is a graphical user interface for \sphinxhref{https://github.com/ashleylst/DSDPy}{DSDPy} package.


\chapter{Tutorial}
\label{\detokenize{tutorial:dsdui-documentation}}\label{\detokenize{tutorial:tutorial}}\label{\detokenize{tutorial::doc}}

\section{Prerequisities}
\label{\detokenize{tutorial:prerequisities}}\begin{itemize}
\item {} 
To create the Anaconda environment:

\end{itemize}

\fvset{hllines={, ,}}%
\begin{sphinxVerbatim}[commandchars=\\\{\}]
\PYG{n}{conda} \PYG{n}{env} \PYG{n}{create} \PYG{o}{\PYGZhy{}}\PYG{n}{f} \PYG{n}{environment}\PYG{o}{.}\PYG{n}{yml}
\PYG{p}{(}\PYG{n}{dsdui}\PYG{p}{)} \PYG{o}{\PYGZgt{}} \PYG{n}{dot} \PYG{o}{\PYGZhy{}}\PYG{n}{c}
\end{sphinxVerbatim}
\begin{itemize}
\item {} 
Manual installation:

\end{itemize}

\fvset{hllines={, ,}}%
\begin{sphinxVerbatim}[commandchars=\\\{\}]
\PYG{n}{conda} \PYG{n}{install} \PYG{o}{\PYGZhy{}}\PYG{n}{c} \PYG{n}{alubbock} \PYG{n}{pysb}
\PYG{n}{conda} \PYG{n}{install} \PYG{o}{\PYGZhy{}}\PYG{n}{c} \PYG{n}{conda}\PYG{o}{\PYGZhy{}}\PYG{n}{forge} \PYG{n}{networkx}
\PYG{n}{conda} \PYG{n}{install} \PYG{o}{\PYGZhy{}}\PYG{n}{c} \PYG{n}{conda}\PYG{o}{\PYGZhy{}}\PYG{n}{forge} \PYG{n}{matplotlib}
\PYG{n}{conda} \PYG{n}{install} \PYG{o}{\PYGZhy{}}\PYG{n}{c} \PYG{n}{anaconda} \PYG{n}{pyqt}
\PYG{n}{conda} \PYG{n}{install} \PYG{o}{\PYGZhy{}}\PYG{n}{c} \PYG{n}{conda}\PYG{o}{\PYGZhy{}}\PYG{n}{forge} \PYG{n}{bidict}
\PYG{n}{conda} \PYG{n}{install} \PYG{o}{\PYGZhy{}}\PYG{n}{c} \PYG{n}{alubbock} \PYG{n}{graphviz} \PYG{n}{pygraphviz}
\end{sphinxVerbatim}


\section{Run program}
\label{\detokenize{tutorial:run-program}}
\fvset{hllines={, ,}}%
\begin{sphinxVerbatim}[commandchars=\\\{\}]
\PYG{n}{python} \PYG{n}{main}\PYG{o}{.}\PYG{n}{py}
\end{sphinxVerbatim}


\section{Usage}
\label{\detokenize{tutorial:usage}}
Upon running the program a standalone application appears:

\begin{figure}[htbp]
\centering
\capstart

\noindent\sphinxincludegraphics{{tutorial}.png}
\caption{DSDPy user interface}\label{\detokenize{tutorial:id1}}\end{figure}


\subsection{Input}
\label{\detokenize{tutorial:input}}\begin{enumerate}
\item {} 
\sphinxstylestrong{DSD model} tab accepts the text input in the form described in \sphinxhref{https://dsdpy.readthedocs.io/en/latest/tutorial.html\#creating-your-own-input}{DSDPy manual}.

\end{enumerate}
\begin{itemize}
\item {} 
choose if the parsing algorithm should be able to: permute the strands, flip the strands, flip the domains

\item {} 
choose if the dots, denoting the destination of the pair, should be displayed

\end{itemize}

\begin{figure}[htbp]
\centering
\capstart

\noindent\sphinxincludegraphics{{dsd_model}.png}
\caption{DSD model tab}\label{\detokenize{tutorial:id2}}\end{figure}
\begin{enumerate}
\setcounter{enumi}{1}
\item {} 
\sphinxstylestrong{Options} tab provides settings for:

\end{enumerate}
\begin{itemize}
\item {} 
threshold of iterations in reaction network generation

\item {} 
rendering speed - the greater the speed, the quicker and less accurate the output rendering (default settings: exponentially increasing for simple species, linearly increasing for pseudoknots)

\item {} 
render button starts the rendering of the current view

\end{itemize}

\begin{figure}[htbp]
\centering
\capstart

\noindent\sphinxincludegraphics{{options}.png}
\caption{Options tab}\label{\detokenize{tutorial:id3}}\end{figure}


\subsection{Simulation}
\label{\detokenize{tutorial:simulation}}
\begin{figure}[htbp]
\centering
\capstart

\noindent\sphinxincludegraphics{{generate_simulate}.png}
\caption{Simulation buttons group}\label{\detokenize{tutorial:id4}}\end{figure}
\begin{enumerate}
\item {} 
\sphinxstylestrong{Generate} button starts the reaction network generation

\item {} 
\sphinxstylestrong{Simulate} button starts the simulation - choose the mode from the combo box (stochastic / deterministic)

\end{enumerate}


\subsection{Output}
\label{\detokenize{tutorial:output}}\begin{enumerate}
\item {} 
\sphinxstylestrong{Input view} tab displays the parsed input species to the DSDPy. \sphinxstylestrong{Output view} tab displays the parsed output species from the DSDPy.

\end{enumerate}
\begin{itemize}
\item {} 
\sphinxstylestrong{Render} button starts the rendering of the current view

\item {} 
choose the input and output of the render from the combo box

\item {} 
save the views as a PNG with save button

\end{itemize}

\begin{figure}[htbp]
\centering
\capstart

\noindent\sphinxincludegraphics{{render}.png}
\caption{DSD species before rendering}\label{\detokenize{tutorial:id5}}\end{figure}

\begin{figure}[htbp]
\centering
\capstart

\noindent\sphinxincludegraphics{{render_done}.png}
\caption{DSD species after rendering}\label{\detokenize{tutorial:id6}}\end{figure}
\begin{enumerate}
\setcounter{enumi}{1}
\item {} 
\sphinxstylestrong{Network} tab displays the chemical reaction network

\end{enumerate}
\begin{itemize}
\item {} 
choose the network layout from the options in the combo box

\item {} 
zoom and pan to navigate through the network

\end{itemize}

\begin{figure}[htbp]
\centering
\capstart

\noindent\sphinxincludegraphics{{network}.png}
\caption{Network tab after clicking Generate button}\label{\detokenize{tutorial:id7}}\end{figure}
\begin{itemize}
\item {} 
click on the species’ name to view the species

\item {} 
click on the reaction name to view the reactants, products and reaction rate of the reaction

\end{itemize}

\begin{figure}[htbp]
\centering
\capstart

\noindent\sphinxincludegraphics{{reaction}.png}
\caption{Reaction window after clicking on a reaction node}\label{\detokenize{tutorial:id8}}\end{figure}
\begin{enumerate}
\setcounter{enumi}{2}
\item {} 
\sphinxstylestrong{Simulation plot} tab displays BNG simulation plot

\end{enumerate}

\begin{figure}[htbp]
\centering
\capstart

\noindent\sphinxincludegraphics{{simulation}.png}
\caption{Simulation tab after clicking Simulate button}\label{\detokenize{tutorial:id9}}\end{figure}
\begin{enumerate}
\setcounter{enumi}{3}
\item {} 
\sphinxstylestrong{Text output} tab displays the text output from the DSDPy

\end{enumerate}

\begin{figure}[htbp]
\centering
\capstart

\noindent\sphinxincludegraphics{{text}.png}
\caption{Text output tab after clicking Generate button}\label{\detokenize{tutorial:id10}}\end{figure}


\chapter{Main modules}
\label{\detokenize{mainmodules::doc}}\label{\detokenize{mainmodules:main-modules}}

\section{elements}
\label{\detokenize{mainmodules:id1}}
Contains wrapper classes for the main concepts of the DSDPy package like \sphinxcode{Domain}, \sphinxcode{Strand} and  \sphinxcode{Species}.


\section{interface}
\label{\detokenize{mainmodules:id2}}
Contains the \sphinxcode{Ui} and \sphinxcode{UiControl} and other classes responsible for the view and control of the user interface.


\section{model}
\label{\detokenize{mainmodules:id3}}
Contains the \sphinxcode{UiModel} class, holding the data and current state of the user interface.


\section{optimization}
\label{\detokenize{mainmodules:id4}}
Contains the logic of the rendering optimization - simulated annealing.


\section{parsing}
\label{\detokenize{mainmodules:id5}}
Contains the input and output files parsers.


\section{utils}
\label{\detokenize{mainmodules:id6}}
Contains the utility functions and the configuration files.


\chapter{Indices and tables:}
\label{\detokenize{index:indices-and-tables}}\begin{itemize}
\item {} 
\DUrole{xref,std,std-ref}{genindex}

\item {} 
\DUrole{xref,std,std-ref}{modindex}

\item {} 
\DUrole{xref,std,std-ref}{search}

\end{itemize}



\renewcommand{\indexname}{Index}
\printindex
\end{document}